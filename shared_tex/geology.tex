Soil characteristics both drive and reflect the uses imposed on it by human activities. Whether a soil is acidic or alkaline, loamy, sandy, clayey, deep or shallow defines its natural functions. It is important to consider these characteristics when a community is considering development and conservation plans. 

Soils function to regulate and filter water flow, decompose vegetative matter and other wastes, provide nutrients for agriculture, and support infrastructure. Once polluted or depleted, soils could require long periods of replenishment/remediation or, at worst, need to be abandoned.

Soil also serves as a habitat for a broad diversity of organisms. Calcareous, or alkaline, soils are often associated with uncommon habitats and rare species. Given their relatively high pH (7.6 to 8.4) they are not ideal for most agricultural uses. 

Recent studies have shown that soils, particularly adjacent to roadways, are vulnerable to salt and other chemical and heavy metal runoff. Depending on drainage characteristics, nearby streams and other bodies of water can also be impacted to the detriment of their dependent organisms. It is appropriate for a town or village to consider alternative cold-weather treatments for road surfaces. Further, stormwater treatment facilities will slow the degradation of soils adjacent to roadways as well. 

Specific soil properties are critical factors to consider in land-use planning: whether it is appropriate or feasible to build, whether septic systems or other types of wastewater treatment can or must be utilized, how much surface area should be left in a permeable state. They can also dictate whether certain building or foundation materials would be subject to corrosion by the pH character of the surrounding soil. It may also be determined that soil is so valuable as an agricultural asset that it should not be subject to commercial or residential development.
