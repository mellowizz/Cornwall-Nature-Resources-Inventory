\label{sec:Wetlands_Why}
Wetlands are areas saturated with surface water or groundwater sufficient to support distinctive vegetation adapted for life in saturated soil conditions. There are many types of wetlands, including those commonly referred to as swamps, marshes, and bogs. They generally have three characteristics in common:
\begin{enumerate}
  \item saturation or flooding for some duration in the growing season,
  \item hydric Soils,
  \item a predominance of hydrophytic (water-loving) vegetation.
\end{enumerate} % 1) (2) hydric soils, and (3) a predominance of hydrophytic (water-loving) vegetation.
%(1) saturation or flooding for some duration in the growing season, (2) hydric soils, and (3) a predominance of hydrophytic (water-loving) vegetation.
Hydric soils and poorly drained soils can indicate the presence of wetlands. They are formed by prolonged saturation that leads to low oxygen conditions and favor the growth of specially adapted plant species. Wetlands are carbon-dense. As they are destroyed, their carbon load is added to the atmosphere.

In addition to providing critical habitat for many plants and animals, wetlands provide important benefits to human communities. They act as a buffer to control flooding and reduce damage from storm surge and provide filtration to cleanse water of impurities. Wetlands hold and slowly release water from sources like snowmelt, flooding, rainfall, or runoff thereby maintaining base streamflow and recharging the groundwater supply that many residents in Cornwall and Cornwall-on-Hudson rely on for a portion of their water supply. Some wetlands are used for recreational purposes as well, like kayaking, canoeing, and birdwatching. 

It is important for municipal planners to be aware of local wetlands in order to proactively plan to conserve this critical resource. While mapping is a useful starting point in identifying wetland areas, small wetlands are often overlooked, so on-site observations are usually necessary to supplement the map information. To fully evaluate how land-use decisions can impact wetlands, adjacent upland areas and connected hydrologic features such as streams must be considered. Hydric soils have been mapped because they are areas where there is a particularly high potential for additional true wetlands.

Because of their critical ecological importance, wetlands are heavily regulated. Their presence is often the reason many municipal (and Private) projects are subjected to an environmental impact review under federal and/or state laws.

To be protected under the Freshwater Wetlands Act, a wetland must be 12.4 acres (5 hectares or larger). Wetlands smaller than this may be protected if they are considered of unusual local importance. Around every wetland is an 'adjacent area' of 100 feet that is also regulated to provide protection for the wetland. The U.S. Army Corps of Engineers (USACE) also protects wetlands, irrespective of size, under Section 404 of the Clean Water Act. Although the USACE definition of wetland is slightly different than the state definition, the Clean Water Act protects basically the same thing -- areas of water or wet soils that support wetland plants.

This map shows four Wetlands designations:
\begin{itemize}
   \item NYS DEC Wetlands (Green)
   \item NWI Wetlands (Purple): The National Wetlands Inventory (NWI) identifies federally designated wetlands.
   \item Probable Wetlands (Hydric Soils) (Pink)
   \item Possible Wetlands (Somewhat Poorly Drained Soils) (Light Brown)
\end{itemize}
