Section: References & Resources

REFERENCES & RESOURCES

References and resources used in the writing of the Cornwall NRI are listed below by the section in which they were referenced. Because Creating a Natural Resources Inventory: A Guide for Communities in the Hudson River Estuary Watershed was referenced throughout, the citation appears only once below.

Haeckel, Ingrid and Laura Heady, Creating a Natural Resources Inventory: A Guide for Communities in the Hudson River Estuary Watershed, Department of Natural Resources, Cornell University, and the New York State Department of Environmental Conservation, Hudson River Estuary Program, Ithaca, NY, 2014.

Historic and Cultural Resources
    • “Here Before Us,” Stony Point Center, www.stonypointcenter.org/get-involved/latest-news/11-latest-news/293-here-before-us-indigenous-history-map-displayed-at-spc. Accessed July 2018. 
    • “History of the Town of Cornwall,” Town of Cornwall, www.cornwallny.com/About-Cornwall/History/History-of-Cornwall. Accessed July 2018.
    • “Environmental Pioneering - Storm King Mountain,” NYS Dept. of Environmental Conservation, www.dec.ny.gov/lands/66618.html. Accessed July 2018. 
    • “Port Jervis Line,” I Ride the Harlem Line, www.iridetheharlemline.com/tag/port-jervis-line/page/3/. Accessed July 2018. 
    • Headley, Russell (Ed.), The History of Orange County, New York, Van Deusen and Elms, 1908.
    • Booth, Malcolm A., A Short History of Orange County N.Y., Greentree Publishing Corp., 1975 (Sponsored by: The Orange County Chamber of Commerce, Inc.).
    • Lifset, Robert D., “Power on the Hudson: Storm King Mountain and the Emergence of Modern American Environmentalism,” University of Pittsburgh Press, 2014.
    • Butzel, Albert K., “Storm King Revisited: A View from the Mountaintop,” Pace Environmental Law Review Vol. 31, Issue 1 (March 2014).

Habitats and Wildlife
	Areas of Known Importance
    • “Coastal Fish and Wildlife Rating Form: Moodna Creek,” NYS Department of State, www.dos.ny.gov/opd/programs/consistency/Habitats/HudsonRiver/Moodna_Creek_FINAL.pdf. Accessed August 2018.
    • Penhollow, Mark E., Paul G. Jensen, and Leslie Zucker, Hudson River Estuary Wildlife and Habitat Conservation Framework, An Approach for Conserving Biodiversity in the Hudson River Estuary Corridor, New York Cooperative Fish and Wildlife Research Unit, Dept. of Natural Resources, Cornell University, NYSDEC, 2006.
    • Edinger, G.J., D.J. Evans, S. Gebauer, T.G. Howard, D.M. Hunt, and A.M. Olivero (editors), Ecological Communities of New York State. Second Edition. A revised and expanded edition of Carol Reschke's Ecological Communities of New York State (draft for review), New York Natural Heritage Program, New York State Department of Environmental Conservation, Albany, NY, 2002.
    • “Opening up the Hudson River for Migrating Fish, One Dam at a Time,” NOAA Office of Response and Restoration, www.response.restoration.noaa.gov/about/media/opening-hudson-river-migrating-fish-one-dam-time.html. Accessed August 2018.
    • “Important Bird Areas: New York,” The Audubon Society, www.audubon.org/important-bird-areas/state/new-york. Accessed August 2018.
    • “Bird Species Checklist,” Black Rock Forest Consortium, www.blackrockforest.org/files/blackrock/content/brf_bird_checklist_current_for_web_0.pdf. Accessed August 2018.

Terrestrial Habitats
    • “Terrestrial Habitat Guides,” Conservation Gateway, The Nature Conservancy, www.conservationgateway.org/ConservationByGeography/NorthAmerica/UnitedStates/edc/reportsdata/hg/terrestrial/Pages/default.aspx. Accessed July 2018. 
    • “Northeast Habitat Map,” Conservation Gateway, The Nature Conservancy, http://maps.tnc.org/nehabitatmap/.
    • “Endangered Species Permits: Habitat Conservation Plans (HCPs) and Incidental Take Permits,” US Fish and Wildlife Service, www.fws.gov/midwest/endangered/permits/hcp/index.html. Accessed July 2018. 
    • “Listed species believed to or known to occur in New York,” US Fish and Wildlife Service, https://ecos.fws.gov/ecp0/reports/species-listed-by-state-report?state=NY&status=listed. Accessed July 2018. 

Forests
    • Bradford Conservation Commission, Natural Resources Inventory: Bradford, New Hampshire, www.bradfordnh.org/documents/BradfordNRI.pdf. Accessed June 2018. 
    • “Hudson Valley,” Lamont-Doherty Earth Observatory, Columbia University Earth Institute, www.ldeo.columbia.edu/tree-ring-laboratory/tree-ring-research/hudson-valley. Accessed June 2018.
    • “Matrix Forest Blocks and Linkages,” NYS Department of Environmental Conservation, http://gis.ny.gov/gisdata/metadata/nysdec.MatrixForestBlocksAndLinkages_Meta.xml. Accessed June 2018,

Stream and Riparian Habitat
    • Esopus Creek News, Upper Esopus Creek Management Plan Newsletter Volume II, Issue III, Cornell University Cooperative Extension Ulster County, Phoenicia, NY, 2007.
    • Moodna Creek Watershed Conservation and Management Plan, Orange County Water Authority, Goshen, NY, 2010.
    • Watershed Design Guide: Best Practices for the Hudson Valley, Orange County Water Authority and Regional Plan Association, Goshen, NY, 2014.

Tidal Wetland Habitat
    • “Tidal Wetland Habitats,” NYS Department of Environmental Conservation, www.dec.ny.gov/lands/87643.html. Accessed June 2018.
    • “Submerged Aquatic Vegetation Habitats,” NYS Department of Environmental Conservation, www.dec.ny.gov/lands/87648.html. Accessed June 2018.
    • “Aquatic Habitats of the Hudson River Estuary,” NYS Department of Environmental Conservation, www.dec.ny.gov/lands/87297.html. Accessed June 2018.

Meadows, Grasslands and Shrublands
    • “Grassland Habitats,” UNH Cooperative Extension, www.extension.unh.edu/resource/grassland-habitats. Accessed June 2018.
    • “New York Grassland Species,” Audubon Society, www.ny.audubon.org/conservation/grasslands. Accessed June 2018.
    • “Protecting Grassland Birds on Private Lands: A Landowner Incentive Program Habitat Protection Project,” NYS Department of Environmental Conservation, www.dec.ny.gov/pubs/32891.html. Accessed June 2018.


Water Resources
Watersheds
    • Moodna Creek Watershed Conservation and Management Plan, Orange County Water Authority, Goshen, NY, 2010.
    • Town of Montgomery Conservation Advisory Council and Town of Wallkill Commission for Conservation of the Environment, Natural Resources Inventory: Towns of Montgomery and Wallkill, Orange County, NY 2010.
    • Town of Rosendale Environmental Commission, Town of Rosendale Natural Resources Inventory, Rosendale, NY, 2010.
    • “The Economic Benefits of Protecting Healthy Watersheds: Factsheet,” United States Environmental Protection Agency, https://www.epa.gov/sites/production/files/2015-10/documents/economic_benefits_factsheet3.pdf. April 2012.
    • “Benefits of Healthy Watersheds,” Healthy Watersheds Protection, United States Environmental Protection Agency, https://www.epa.gov/hwp/benefits-healthy-watersheds. Accessed August 2018.
    • Orange County Water Authority, http://waterauthority.orangecountygov.com/index.html. Accessed August 2018.

Groundwater and Aquifers
    • “Environmental Site Database Search,” New York State Department of Environmental Conversation, https://www.dec.ny.gov/chemical/8437.html. Accessed April 2018.
    • “Infiltration – The Water Cycle,” The Water Cycle – USGS Water Science School, United States Geological Survey, https://water.usgs.gov/edu/watercycleinfiltration.html. Accessed August 2018.
    • Orange County Water Authority, Moodna Creek Watershed Atlas, Orange County, NY, 2008.
    • “Source Water Protection,” New York Rural Water Association, http://www.nyruralwater.org/technical-assistance/source-water-protection. Accessed August 2018.
    • Town of Montgomery Conservation Advisory Council and Town of Wallkill Commission for Conservation of the Environment, Natural Resources Inventory: Towns of Montgomery and Wallkill, Orange County, NY 2010.
    • Town of Rosendale Environmental Commission, Town of Rosendale Natural Resources Inventory, Rosendale, NY, 2010.

Floodplains
    • “Floodplain Management,” HUD Exchange, US Department of Housing and Urban Development, https://www.hudexchange.info/environmental-review/floodplain-management/. Accessed April 2018.
    • “Floodplain Management,” NYS Department of Environmental Conservation, http://www.dec.ny.gov/lands/24267.html. Accessed April 2018.
    • “Flood Risk Management Program,” United States Army Corps of Engineers, https://www.iwr.usace.army.mil/Missions/Flood-Risk-Management/Flood-Risk-Management-Program/. Accessed April 2018.
    • “Floods: Recurrence Intervals and 100-Year Floods,” United States Geological Survey, https://water.usgs.gov/edu/100yearflood.html. Accessed April 2018
    • “Flood Zones,” Federal Emergency Management Agency, https://www.fema.gov/flood-zones. Accessed August 2018.
    • “Identify Problems,” New York Climate Change Science Clearinghouse, www.nyclimatescience.org/highlights/problems#downpours. Accessed January 2018.
    • Lind, Dara, “The “500-year flood” explained: why Houston was so underprepared for Hurricane Harvey,” Vox, August 28, 2017.

Wetlands
    • “Carbon Sequestration,” Association of State Wetland Managers, https://www.aswm.org/wetland-science/wetlands-and-climate-change/carbon-sequestration. Accessed September 2018.
    • New York State Department of Environmental Conservation, Division of Fish, Wildlife and Marine Resources, Article 24, Freshwater Wetlands, Title 23 of Article 71 of the Environmental Conservation Law, reprinted May 1997.
    • United Stated Army Corps of Engineers, Recognizing Wetlands, 1998.
    • United Stated Environmental Protection Agency, Functions and Values of Wetlands, September 2001.
    • “Vernal Pools,” United Stated Environmental Protection Agency, https://www.epa.gov/wetlands/vernal-pools. Accessed September 2018.
    • “Wetlands Identification,” United Stated Army Corps of Engineers – New York District, http://www.nan.usace.army.mil/Missions/Regulatory/Wetlands-Identification/. Accessed September 2018.

Streams and Water Quality
    • “Biomonitoring,” New York State Department of Environmental Conservation, http://www.dec.ny.gov/chemical/23847.html.  Accessed April 2018.
    • “Lower Hudson River WI/PWL,” New York State Department of Environmental Conservation, https://www.dec.ny.gov/chemical/36740.html. Accessed September 2018.
    • Orange County Water Authority, Moodna Creek Watershed Conservation and Management Plan, Goshen, NY, 2010.
    • “Fact Sheet on Assessment of Water Quality Impact on Streams and Rivers,” New York State Department of Environmental Conservation, http://www.dec.ny.gov/docs/water_pdf/bapnarrative18.pdf. Accessed September 2018.
    • New York State Department of Environmental Conservation, Final New York State 2016 Section 303(d) List of Impaired Waters, November 2016.
    • New York State Department of Environmental Conservation, Division of Water, Standard Operating Procedure: Biological Monitoring of Surface Waters in New York State, April 2014.
    • New York State Department of Environmental Conservation, The Lower Hudson River Basin, 2006.
    • New York State Department of Environmental Conservation, WI/PWL Fact Sheet – Moodna Creek/Hudson River, July 2008.
    • “Rotating Integrated Basin Studies (RIBS),” New York State Department of Environmental Conservation, http://www.dec.ny.gov/chemical/30951.html. Accessed September 2018.
    • “State Pollutant Discharge Elimination System Compliance Assurance Program,” NYWEA Clear Waters, New York State Department of Environmental Conservation, Summer 2010, http://www.dec.ny.gov/chemical/67777.html. Accessed April 2018.
    • “Water Basics Glossary,” Water Resources of the United States, United States Geological Survey, https://water.usgs.gov/water-basics_glossary.html#T. Accessed September 2018.
    • “Water Quality Information,” New York State Department of Environmental Conservation,” http://www.dec.ny.gov/chemical/8459.html. Accessed April 2018.

Geology and Soils
Bedrock Geology
    • “Bedrock, Geology,” Encyclopedia Britannica, www.britannica.com/science/bedrock. Accessed July 2018.
    • McGoey, Hauser and Edsall Consulting Engineers, P.C., Regional Ground-water Study, Town of Cornwall, Orange County, NY. Orange County Water Authority, 1994.
    • Anderson MG, Ferree CE (2010), “Conserving the Stage: Climate Change and the Geophysical Underpinnings of Species Diversity,” PLoS ONE 5(7): e11554. doi:10.1371/journal.pone.0011554.

Steep Slopes
    • “Steep Slopes/Slope Analysis,” Natural Resource Inventory (NRI), Franklin Township, Warren County, NJ, www.franklintwpwarren.org/InventoryPDFs/chap04.pdf. Accessed June 2018. 
    • “Steep Slopes In Chatham Township,” Chatham Township Environmental Commission, Chatham NJ, www.chathamtownship-nj.gov/images/CTEC/NRI1999/slopesadd090704.pdf. Accessed June 2018.
    • “Steep Slope Ordinance,” ConservationTools.org, Pennsylvania Land Trust Association, www.conservationtools.org/guides/59. Accessed June 2018.
    • Town of Rosendale Environmental Commission, Town of Rosendale Natural Resources Inventory, Rosendale, NY, 2010.

Soils
    • “Management of Calcareous Soils,” The Food and Agriculture Organization (FAO) of the United Nations, www.fao.org/soils-portal/soil-management/management-of-some-problem-soils/calcareous-soils/en/. Accessed August 2018.
    • “Soil Survey of Orange County,” U.S. Dept. of Agriculture Natural Resources Conservation Service, www.nrcs.usda.gov/Internet/FSE_MANUSCRIPTS/new_york/NY071/0/orange.pdf. 1981.
    • “Prime and Important Farmlands in New York,” U.S. Dept. of Agriculture Natural Resources Conservation Service, https://efotg.sc.egov.usda.gov/references/public/NY/Farmland_Class_NY_Information.pdf. Accessed June 2018.

Climate Conditions and Projections
    • “Aquifers and Groundwater,” US Geological Society, https://water.usgs.gov/edu/earthgwaquifer.html. Accessed April 2018.
    • “Climate Explorer – Orange County, NY,” U.S. Climate Resilience Toolkit, https://toolkit.climate.gov/climate-explorer2/location.php?county=Orange+County&city=Cornwall,%20NY&fips=36071&lat=41.4020757&lon=-74.04176280000001#location-temperature. Accessed January 2018.
    • “Climate Impacts on Human Health,” United Stated Environmental Protection Agency, https://19january2017snapshot.epa.gov/climate-impacts/climate-impacts-human-health_.html. Accessed January 2018.
    • Climate Smart Communities Certification Actions, Climate Smart Communities Program, New York State Department of Environmental Conservation. Accessed September 2018.
    • “Compare Counties,” AirCompare, https://www3.epa.gov/aircompare/#trends. Accessed January 2018.
    • “Criteria Air Pollutant Descriptions,” Scorecard: The Pollution Information Site, http://scorecard.goodguide.com/env-releases/cap/pollutant-desc.tcl#EDF-213. Accessed January 2018.
    • “Education – Aquifer,” National Geographic Society, https://www.nationalgeographic.org/encyclopedia/aquifer/. Accessed April 2018.
    • Environmental Protection Bureau of the New York State Attorney General, Current & Future Trends in Extreme Rainfall across New York State, September 2014.
    • Horton, R., D. Bader, C. Rosenzweig, A. DeGaetano, and W.Solecki, Climate Change in New York State: Updating the 2011 ClimAID Climate Risk Information, Supplement to NYSERDA Report 11-18 (Responding to Climate Change in New York State), New York State Energy Research and Development Authority (NYSERDA), Albany, New York, 2014.
    • “Identify Problems,” New York Climate Change Science Clearinghouse, www.nyclimatescience.org/highlights/problems#downpours. Accessed January 2018.
    • “Interactive Map and GIS Viewer,” New York Climate Change Science Clearinghouse, https://www.nyclimatescience.org/catalog/doc?DocId=vitroIndividual:http://www.nyclimatescience.org/individual/n7494. Accessed January 2018.
    • Melillo, Jerry M., Terese (T.C.) Richmond, and Gary W. Yohe, Eds., Climate Change Impacts in the United States:  The Third National Climate Assessment, U.S. Global Change Research Program, Washington, DC, 2014.
    • Mid-Hudson Regional Economic Development Council, Strategic Plan, 2017.
    • “New York’s Changing Climate,” Climate Change Facts, prepared by Dr. Art DeGaetano et al. for Cornell University College of Agriculture and Life Sciences, Cornell Cooperative Extension, October 2011.
    • New York State Climate Smart Communities, Climate Smart Resiliency Planning: A Planning Evaluation Tool for New York State Communities, Version 2.0, October 2014.
    • “Orange County, NY Environmental Health Statistics,” Healthgrove, http://environmental-health.healthgrove.com/l/1918/Orange-County-NY#Air%20Pollution&s=36oJBl. Accessed January 2018.
    • “The Earth’s Changing Climate,” Climate Change Facts, prepared by Dr. Art DeGaetano et al. for Cornell University College of Agriculture and Life Sciences, Cornell Cooperative Extension, September 2011.
    • Zemaitis, Libby, “Working Toward Climate Resilience: General Climate Information Prepared for Hudson Valley Communities,” Hudson River Estuary Program/Cornell University, New York State Department of Environmental Conservation, New Paltz, NY, 2018.

Fossil Fuel Industry
(Niklas: Not sure if your references will apply to both sections, so may not need to be separated.)

Pilgrim Pipeline

Anchorage Barges

Land Use
Land Use and Land Cover
    • C-CAP (Coastal Change Analysis Program) Land Cover Atlas, National Oceanic and Atmospheric Administration (NOAA), https://coast.noaa.gov/digitalcoast/tools/lca.html. Accessed May 2018.
    • Land Use Law Center of Pace University Law School, Beginner’s Guide to Land Use Law, White Plains, NY.
    • “Smart Growth Principles,” Smart Growth Online, Smart Growth Network, https://smartgrowth.org/smart-growth-principles/. Accessed May 2018.
    • Schueler, T. R., “The Importance of Imperviousness,” Watershed Protection Techniques, 1(3): 100-111, 2000.
    • National Land Cover Database, “United States Geological Survey,” https://www.mrlc.gov/. Accessed May 2018.

Farmland 
    • American Farmland Trust, https://www.farmland.org/our-work/where-we-work/new-york.
    • “NY Ag Facts,” New York State Agricultural Society, www.nysagsociety.org/ny-ag-facts. Accessed August 2018.
    • “New York State Agriculture Snapshot,” New York State Department of Agriculture and Markets, www.agriculture.ny.gov. Accessed August 2018. 
    • “Property Type Classification Codes - Assessors' Manual,” NYS Department of Taxation, www.tax.ny.gov/research/property/assess/manuals/prclas.htm. Accessed August 2018. 
    • Ditzler, C., K. Scheffe, and H.C. Monger (Eds.). Soil Survey Manual: USDA Handbook 18. U.S. Dept. of Agriculture Natural Resources Conservation Service Soil Science Division Staff, Government Printing Office, Washington, D.C., 2017.
    • “2017 State Agriculture Overview: New York,” U.S. Dept. of Agriculture, National Agricultural Statistics Service, www.nass.usda.gov/Quick_Stats/Ag_Overview/stateOverview.php?state=NEW%20YORK. Accessed August 2018.

Conservation and Public Lands
Decker, Matt, Director of Conservation and Stewardship, Orange County Land Trust. 

Zoning and Tax Maps
    • Code of the Town of Cornwall. Accessed April 2018.
    • Chapter 75: Clearing and Grading, https://ecode360.com/10555069.
    • Chapter 90: Freshwater Wetlands, https://ecode360.com/10555547.
    • Chapter 121: Stormwater Management, https://ecode360.com/14141956.
    • Chapter 158: Zoning, https://ecode360.com/10556572.
    • Code of the Village of Cornwall-on-Hudson. Accessed April 2018.
    • Chapter 83: Flood Damage Prevention, https://ecode360.com/15449238.
    • Chapter 132: Stormwater Management, https://ecode360.com/15450043.
    • Chapter 151: Trees, Shrubs, and Bushes, https://ecode360.com/15451101.
    • Chapter 168: Wetlands, Freshwater, https://ecode360.com/15451501.
    • Chapter 172: Zoning, https://ecode360.com/15448596.
    • Strong, Karen, Conserving Natural Areas and Wildlife in Your Community: Smart Growth Strategies for Protecting the Biological Diversity of New York’s Hudson River Valley, New York Cooperative Fish and Wildlife Research Unit, Cornell University, and the New York State Department of Environmental Conservation, Hudson River Estuary Program, Ithaca, NY, 2008.
    • Hudson Valley Regional Council, Protecting and managing Hudson River streams: The Importance of stream buffer protection and management, November 2015.
    • Orange County Water Authority, Model Riparian Buffer Local Law. Accessed May 2018.
    • Orange County Water Authority and Regional Plan Association, Municipal Resiliency Code Audit, 2014.
    • Orange County Water Authority and Regional Plan Association, Watershed Design Guide: Best Practices for the Hudson Valley. Accessed May 2018.
    • Salomone, Kory, editor, Starting Ground Series, Gaining Ground: Training Book for Land Use Leaders, Land Use Law Center, Pace Law School, White Plains, NY, 2004.
    • “Recognizing Wetlands,” US Army Corps of Engineers, New York District, http://www.usace.army.mil/Portals/2/docs/civilworks/regulatory/rw_bro.pdf. Accessed April 2018.
    • “US Army Corps of Engineers, New York District,” Wetlands Identification, http://www.nan.usace.army.mil/Missions/Regulatory/Wetlands-Identification/. Accessed April 2018.