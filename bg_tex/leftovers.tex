%-----------------------------------------------------------

%% two pages per Map
%% QR Code
%% Glossary
%% Acknowledgements
%% Rosendale format
%% Map
%% 1.08 line spacing
%% .75in margin
%% new lebanon NRI
%% section
%% General section: %% Why is this map important?
%%%% how 
%% How to use?
%% section headings
%% subsections
%% Protected open spaces?

%% Header with subsection/section name
%% Two different

\subsection{Bedrock Geology}

\subsection{Calcerous and Glacial outwash Soils}
\subsection{Slopes}

\subsection{Introduction}
A region's landscape has been shaped by thousands of years of geologic activity and greatly influences where humans settle. Settlements are generally located on level ground where freshwater is plentiful. If no freshwater surface water is available, places with sand and gravel are ideal as groundwater is more readily accessible in areas with such geology. The geology also determines soil characteristics and what mineral resources are available. Today, taking geology into consideration is important when finding suitable places for infrastructure such as roads and buildings, septic systems, wells and productive agricultural areas to name a few. Furthermore, understanding a region's geology and the processes that change it, is vitally important for sound resource management as a region's quality and quantity of groundwater, flooding and drainage considerations, soil and environmental hazards are all influenced by its geology.

\subsection{Geologic History}
New York's unique geology was formed over many years of glacial retreat. Shale is New York's most common sedimentary rock that forms from mud and clay \cite{vandiver}.

Surficial geology is the study of landforms and the geologic 
materials lying on top of the bedrock. These materials can be sand and gravel, clay and silts, and glacial tills.

Cornwall is in the Hudson Highlands which is made up of mostly gneisses that were formed in the Precambrian period. Cornwall is unique in that it is adjacent to the Hudson fjord, which is a "steep-sided, coastal valley first carved by a stream, then by a glacier, then drowned by the sea" ~\citep{vandiver}, 87.

While 31.5\% of Orange County's geology is the Austin Glen Formation, the Hudson Highlands, which Cornwall is a part of is mostly made up of metamorphic rocks named gneisses. The Austin Glen Formation's primary rock is graywacke, a grayish sandstone and the secondary rock is shale which is interbedded in the graywacke.

Cornwall and Blooming Grove are no 
different as shale mines are still in operation in Blooming Grove and 
sand. Gravel mines (the last in Cornwall closed 2009) used to operate within the 

Cornwall has its share of stunning magnificent granite outcroppings in Storm King that have captured the imagination of settlers and tourists over the years. The town is underlayed by Basalt, which is formed by cooling basaltic lava. Basalt, the most common rock found in continental areas, is also the main component of the ocean floor. 

The Town's eastern end consists of bedrock within the Mount Merino formation and the
Austin Glen formation (Graywacke shale). The surface of the bedrock varies in depth from surface exposure to greater than 100 feet deep.

A report by the \gls{ocwa} found that wells drilled into 
bedrock units within the Town are not highly productive. Most residential wells 
within the Town are low yield bedrock wells. No high yield bedrock wells were 
identified during this study.

The western part of town where Shunnemunk Ridge is located near Blooming Grove 
is mostly shale and siltstone.

The Village of Cornwall on Hudson has two main geologic features bisecting the 
Village: Granite gneiss and Graywake shale. There is a small band of Mafic gneiss in the southwest corner of the Village.

\subsection{Bedrock and Surficial Geology}
There are three active mines in Blooming Grove, with a combined area of slightly more than 43 acres.
There were three sand and gravel mines that are now inactive.

16.4\% of the soils within the study area are Mardin gravelly silt loam with 3 to 8 percent slopes. 

\subsection{Cornwall specific}
The southeast portion of the Town of Cornwall is composed of undifferentiated Gneiss, Granite and Granitic Gneiss. Storm King Mountain is primarily composed of granite (granitic gneiss) and a small band of Mafic gneiss. The Town's southwest, where the Skunnemunk Mountain State Park is located, consists of layered sandstone and Martinsburg formation (shale). The Skunnemunk conglomerate formation overlays the Skunnemunk ridge (OCWA, 1994). Sedimentary bedrock makes up the rest of town; made up of the following units:

\begin{enumerate}
\item Undifferentiated Hamilton Group (shale, siltstone)
\item Martinsburg formation (shale, siltstone)
\item Undifferentiated Lower Devonion and Silurian Rocks (sandstone, shale, conglomerate)
\item Mount Merino Formation 
\end{enumerate}
(OCWA, 1994)

\begin{table}[h!]
    %\centering
    \begin{tabular}{l c c c}
    Unit Name & Acres & Percentage \\
    \hline
    Hornblende granite and granite gneiss &	7300.5 & 41\%\\
    Austin Glen Formation (Pawlet in Vermont) &	3909.0 & 22\% \\
    Glacial and Alluvial Deposits &	2188.4 & 12\% \\
    Undifferentiated Hamilton Group & 1831.2 & 10\% \\
    Pyroxene-hornblende-quartz-plagioclase gneiss & 1257.8 & 7\%\\
    Water & 584.7 & 3\%\\
    Wappinger Group & 515.3 & 3\% \\
    Cambrian thru Middle Ordovician carbonate rock & 136.9 & <1\%\\
    Garnet-biotite-quartz-feldspar gneiss &	111.4 & < 1\%\\
    \end{tabular}
    \caption{Lithography}
    \label{tab:Cornwall_lith}
\end{table}

\subsection{Blooming Grove}
Blooming Grove is made up of 
limestone

There is a stripe of shale running southwest in Blooming Grove. 

\subsection{Soils}
\subsection{Slopes}


\subsection{Forest Cover}

%%%%%% Inter municipal %%%%%%%%%%%%%%%
\subsection{Forest Patches and Linkage Zones}
\subsection{Important Ecological Areas}
\subsection{Protected Open Space}
\subsection{Watersheds and Sub-basins}

%%%%%%%%%%%%%%%%%%%%%%%%%%%%%%%%%%%%%%
NYS Regulatory Freshwater Wetlands
\begin{table}
\begin{tabular}{p{7cm} p{7cm}}
\label{tab:culverts}
Municipality & number of culverts \\
\hline
Blooming Grove & 13\\
Cornwall & 18\\
\acrlong{coh} & 2\\
\end{tabular}
\end{table}

\subsection{Protected Lands}
The Nature Conservancy has an approximately 292 acre parcel on the west side of the Shawnagunk Mountain Ridge and south of Otterkill Rd.