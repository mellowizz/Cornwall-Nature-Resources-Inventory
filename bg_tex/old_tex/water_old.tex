



\subsection{Floodplains and Flooding}
\subsection{Biomonitoring and Priority Waterbodies}
\subsection{Stream Classification}
Without drinking water, the Town cannot survive. 

Besides this most important need of the \gls{bg}, creeks, kills, streams, ponds, lakes, and their adjacent riparian (streamside) habitats provide many benefits to communities, such as flood management, and recreational opportunities. The creeks that originate in The Town of Blooming Grove merge to form the Moodna Creek.  The Moodna Creek then flows through the Village of Washingtonville, the Towns of New Windsor and Cornwall and into the Hudson River. 

For further information refer to the Stream Biomonitoring and Priority Water Bodies map and documentation.
\includepdf[landscape=true, noautoscale=true, fitpaper=true]{maps/BG_Stream_Monitoring_Priority_Water_Bodies}
%\include{maps/BG_Stream_Monitoring_Priority_Water_Bodies}
The Biological Assessment Profile (BAP) Scores provided by the Orange County Water Authority are a method of plotting biological index values on a common scale of water quality impact. Values from five indices are converted to a common 0-10 scale.  

The Biological Assessment Profile (BAP) Scores on the map show: 

AA or A - Not Impacted, very good water quality 

B - Slightly Impacted, good water quality, good fish survival but may not be good for fish propagation 

C or CT - Moderately Impacted (orange), poor water quality, fish propagation but usually not to fish survival

D – Lowest class 
\subsection{Wells and Risk Sites}
\subsection{Wetlands and Hydric Soils}
\label{sec:Wetlands_Why}
Wetlands are areas saturated with surface water or groundwater sufficient to support distinctive vegetation adapted for life in saturated soil conditions. There are many types of wetlands, including those commonly referred to as swamps, marshes, and bogs. They generally have three characteristics in common:
\begin{enumerate}
  \item saturation or flooding for some duration in the growing season,
  \item hydric Soils,
  \item a predominance of hydrophytic (water-loving) vegetation.
\end{enumerate} % 1) (2) hydric soils, and (3) a predominance of hydrophytic (water-loving) vegetation.
%(1) saturation or flooding for some duration in the growing season, (2) hydric soils, and (3) a predominance of hydrophytic (water-loving) vegetation.
Hydric soils and poorly drained soils can indicate the presence of wetlands. They are formed by prolonged saturation that leads to low oxygen conditions and favor the growth of specially adapted plant species. Wetlands are carbon-dense. As they are destroyed, their carbon load is added to the atmosphere.

In addition to providing critical habitat for many plants and animals, wetlands provide important benefits to human communities. They act as a buffer to control flooding and reduce damage from storm surge and provide filtration to cleanse water of impurities. Wetlands hold and slowly release water from sources like snowmelt, flooding, rainfall, or runoff thereby maintaining base streamflow and recharging the groundwater supply that many residents in Cornwall and Cornwall-on-Hudson rely on for a portion of their water supply. Some wetlands are used for recreational purposes as well, like kayaking, canoeing, and birdwatching. 

It is important for municipal planners to be aware of local wetlands in order to proactively plan to conserve this critical resource. While mapping is a useful starting point in identifying wetland areas, small wetlands are often overlooked, so on-site observations are usually necessary to supplement the map information. To fully evaluate how land-use decisions can impact wetlands, adjacent upland areas and connected hydrologic features such as streams must be considered. Hydric soils have been mapped because they are areas where there is a particularly high potential for additional true wetlands.

Because of their critical ecological importance, wetlands are heavily regulated. Their presence is often the reason many municipal (and Private) projects are subjected to an environmental impact review under federal and/or state laws.

To be protected under the Freshwater Wetlands Act, a wetland must be 12.4 acres (5 hectares or larger). Wetlands smaller than this may be protected if they are considered of unusual local importance. Around every wetland is an 'adjacent area' of 100 feet that is also regulated to provide protection for the wetland. The U.S. Army Corps of Engineers (USACE) also protects wetlands, irrespective of size, under Section 404 of the Clean Water Act. Although the USACE definition of wetland is slightly different than the state definition, the Clean Water Act protects basically the same thing -- areas of water or wet soils that support wetland plants.

This map shows four Wetlands designations:
\begin{itemize}
   \item NYS DEC Wetlands (Green)
   \item NWI Wetlands (Purple): The National Wetlands Inventory (NWI) identifies federally designated wetlands.
   \item Probable Wetlands (Hydric Soils) (Pink)
   \item Possible Wetlands (Somewhat Poorly Drained Soils) (Light Brown)
\end{itemize}

Blooming Grove’s geology created the conditions for wetlands, hydric and poorly drained soils. The steeper and more rugged southeast of Blooming Grove from Woodcock Hill in the north to Shunnemunk mountain in the south has fewer wetlands, hydric soils and poorly drained soil than the rest of the Town. These areas have two different bedrock geologies: a mixture of shale with siltstone and limestone. The Town has slightly fewer than 1,438 acres of wetlands according to the National Wetland Inventory and approximately 1,861 acres of DEC regulated wetlands intersect the town. Of those 1,254 are within the town. To put that into perspective, the Town of Blooming Grove contains 1\% of the DEC regulated wetlands in Orange County. Given that Blooming Grove is a little over 4\% of the area of Orange County and one out of thirty-two municipalities, it has less than half its share of wetlands if they were equally distributed and all municipalities were the same size. If one were to include the hydric soils, the percentage of wetlands and probable wetlands increases. One notices that the west of the Town bounded by N.Y. Rte. 94 to the north and west of County Rte. 51, has a few large wetlands.
According to DEC data from 2011, 6\% of the County is wetland. Over three quarters of the DEC wetlands in Blooming Grove are classified as “class II” wetlands. While class I wetlands are those wetlands which may contain endangered or threatened species, the class II designation may have any of 17 features and include endangered or threatened migratory species or resident species that are vulnerable among other criteria.

The most densely populated area of Blooming Grove, the Village of Washingtonville, has quite a large percentage of wetlands and hydric soils. The poorly drained soils in Washingtonville appear to have housing developments built on them adding to the flooding risk. The amount of sealed surface probably exacerbated the flooding during Hurricane Irene in 2011. When not built upon and sealed by road surfaces, wetlands and hydric soils help mitigate flooding.

%\section{Streams and Water Bodies}
%\subsection{Background}
%%%% culverts %%%%
%%%%% statistics %%%%%%
\label{tab:bg_wetlands}
\begin{table}[h!]
    %\centering
    \begin{tabular}{l l c c }
    Name & Acres & Percentage \\
    \hline
    Freshwater Forested/Shrub Wetland & 191  & 31\%  &\\
   	Lake & 166 & 27\% &\\ 
    Freshwater Pond & 136 & 22\% &\\
    Riverine & 38 & 11\% &\\
    Freshwater Emergent Wetland & 86 & 14 \% & \\
    total & 617 & 100\% &\\
    \end{tabular}
    \caption{National Wetland Inventory}
    \label{tab:Cornwall_nwi}
\end{table}

\begin{table}[h!]
    %\centering
    \begin{tabular}{l l c c }
    Name & Acres & Percentage \\
    \hline
    Freshwater Pond & 4 & 36\% &\\
    Freshwater Emergent Wetland & 4 & 33\% &\\
    Riverine & 3 & 22\% &\\
    total & 11 & 100\% &\\
    \end{tabular}
    \caption{\gls{coh} National Wetland Inventory}
    \label{tab:Coh_nwi}
\end{table}

%%%%%%%%%%%%%%%%% water bodies %%%%%%%%%%%%%
%The town of \acrlong{bg} has of water bodies. 
%The town of Cornwall has ____ of water bodies.

\subsection{Public wells and Aquifers}

\subsection{Culverts and Stormwater management}
%% insert image of flooded roads
22919m of road were flooded in the area of interest.
In Cornwall 1976.3 m of roadway were flooded and 18691.9m in Blooming Grove. There are 28 bridges in the town of Cornwall. There are 26 in the Town of Blooming Grove.
\begin{table}
    %\centering
    \begin{tabular}{l c } 
    Municipality & \\
    \hline
    Blooming Grove & 1 \\
    Cornwall & \\
    \end{tabular}
    \caption{Flooded roads in 2007}
    \label{tab:flooded_roads}
\end{table}

%%%% culverts %%%
\begin{table}
\begin{tabular}{p{7cm} p{7cm}}
\label{tab:culverts}
Municipality & Number\\
\hline
Blooming Grove & 13\\
Cornwall & 18\\
\acrlong{coh} & 2\\
\end{tabular}
\end{table}

Wetlands provide a wide array of scale-dependent ecosystem services that are 
increasingly being recognized and appreciated by the public and decision-makers 
~\citep{poschlod2007}. Locally and regionally, wetlands provide carbon and nutrient 
cycling, flood management, water storage (e.g. after heavy rainfall or 
snowmelt) and filtering, cultural (e.g. sources of creativity), soil buffer 
(e.g. heavy metals) as well as habitat services ~\citep{poschlod2007}. ``...those 
areas, inundated or saturated by surface or ground water at a frequency and 
duration sufficient to support, and that under normal circumstances do support, 
a prevalence of vegetation typically adapted for life in saturated soil 
conditions.“ (40 CFR 232.2(r)) \\ %newline. 
According to the USDA's 
Engineering Field Handbook on Wetland Restoration, Enhancement, or Creation of 
Wetlands, for the purpose of this chapter, are defined as areas that have 
an-aerobic soil conditions due to the presence of water, at or near the surface 
for a sufficient duration to support wetland vegetation."

Poorly planned development can increase runoff, chemicals, sediment, and other 
contaminants entering streams and waterbodies, threatening water quality, 
degrading habitat value, and increasing flood risk. Precipitation has become 
more variable and extreme with climate change in the Northeast, exacerbating 
these threats. Annual rainfall occurring in heavy downpours increased 74\% 
between the periods of 1950-1979 and 1980-2009, and most areas of the Hudson 
Valley have been impacted by serious flooding in recent years ~\citep{rosenzweig2011climate} 