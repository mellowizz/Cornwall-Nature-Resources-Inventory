Soil characteristics both drive and reflect the uses imposed on it by human activities. Whether a soil is acidic or alkaline, loamy, sandy, clayey, deep or shallow defines its natural functions. It is important to consider these characteristics when a community is considering development and conservation plans. 

Soils function to regulate and filter water flow, decompose vegetative matter and other wastes, provide nutrients for agriculture, and support infrastructure. Once polluted or depleted, soils could require long periods of replenishment/remediation or, at worst, need to be abandoned.

Soil also serves as a habitat for a broad diversity of organisms. Calcareous, or alkaline, soils are often associated with uncommon habitats and rare species. Given their relatively high pH (7.6 to 8.4) they are not ideal for most agricultural uses. 

Recent studies have shown that soils, particularly adjacent to roadways, are vulnerable to salt and other chemical and heavy metal runoff. Depending on drainage characteristics, nearby streams and other bodies of water can also be impacted to the detriment of their dependent organisms. It is appropriate for a town or village to consider alternative cold-weather treatments for road surfaces. Further, stormwater treatment facilities will slow the degradation of soils adjacent to roadways as well. 

Specific soil properties are critical factors to consider in land-use planning: whether it is appropriate or feasible to build, whether septic systems or other types of wastewater treatment can or must be utilized, how much surface area should be left in a permeable state. They can also dictate whether certain building or foundation materials would be subject to corrosion by the pH character of the surrounding soil. It may also be determined that soil is so valuable as an agricultural asset that it should not be subject to commercial or residential development.

Slightly more than three quarters of the bedrock in the Town is part of the
Austin Glen Formation and composed of primarily graywacke and shale, which are
sedimentary types of carbonate rock. Given that there are still active shale
mines in Blooming Grove this does not come as a surprise. The first name in the
legend, in this case “Middle Ordovician” denotes what time period the geology
was formed. The formation is potentially calcareous, meaning it is mostly or
partly made up of calcium carbonate. The other large formations, at twelve and
seven percent respectively are Undifferentiated Hamilton Group and
Undifferentiated Lower Devonian and Silurian rocks. The former is mostly
comprised of shale and siltstone and the latter is composed of limestone and
dolostone. Siltstone is also another sedimentary rock and contains more quartz
than either mudstones or shales. The rock is formed by silt grade sediment in
either marine or freshwater environments being compacted. Limestone is rock
that is made up of mostly calcium carbonate. At just under five percent,
Garnet-biotite-quartz-feldspar gneiss is composed of gneiss and quartzite and
is the only other rock type that covers more than one percent of the town’s
area.

\subsection{soils}

\begin{table}[h!]
    %\centering
    \begin{tabular}{l l c c }
    Name & Acres & Percentage \\
    \hline
    MdC & 5665 & 18.50\%\\
    ErB & 1846 & 6.03\%\\
    RSD & 1804 & 5.89\%  \\
    RSB & 1242 & 4.05\%\\
    Ab & 1009 & 3.29\%\\
    SXC & 914 & 2.98\%\\
    ANC & 870 & 2.84\%\\
    ANF & 840 & 2.74\% \\
    total & 20243 & 66\% &\\
    \end{tabular}
    \caption{Blooming Grove Soil Classes}
    \label{tab:BG_soil}
\end{table}