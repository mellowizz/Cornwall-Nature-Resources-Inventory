\subsection*{Why you need this map}
A base map depicts background reference information such as roads, landmarks, 
political boundaries, and landforms, onto which the other thematic information 
of this NRI is displayed. The base map also provides a visual reference of 
areas of residential and commercial development, as well as important municipal 
features. The base map includes the entire town and a one-mile extension to 
show the resources that extend beyond the municipal borders for all maps. The 
Town of Cornwall has a land area of 26.65 square miles based on 2010 US Census 
Bureau data. 
\par
The two aerial imagery maps of the town are also helpful for general 
orientation. They display two distinct views of the town and village eight 
years apart, and differentiated by presence and absence of seasonal vegetation. 
The 2007 orthoimagery was taken in the early spring using a DMC sensor flown at 
a nominal height of 4500 feet \gls{amt}. The 2016 orthoimagery was taken in the 
summer using a Microsoft Ultracam Eagle sensor flown at a nominal height of 
7400 feet AMT. The pixel sizes are half a foot for both natural color and color 
infrared images. The resolution is listed as being 4ft. horizontally at 95\% 
confidence interval for true 1ft. resolution. New York State started offering 
1ft. resolution after 2014.
\par
A comparison of the two aerial images shows very little change to the general 
characteristics of the town’s development, open spaces, and forest cover, with 
the exception of intermittent residential development along State Rt. 32 and 
large-scale development along the north side of Rt. 9W into one of the few 
remaining large parcels of unprotected stepping stone forest in the town. 

\subsection*{Base Map}\label{subsec:basemap}
The Town of Cornwall includes the Village of Cornwall-on-Hudson. The Town of 
New Windsor borders Cornwall to the north, the Towns of Highlands and Woodbury 
border to the south, and the Town of Blooming Grove borders to the west. The 
base map includes key road names, route numbers, and municipal authority 
designations as well as and other transportation infrastructure. Also included 
are important municipal structures and notable natural features like bodies of 
water and elevational topography

\subsection*{Transportation networks}
\begin{itemize}
    \item Interstate Route 87 
    \item US (Federal) Route 9W
    \item NY State Routes 94, 32, and 218
    \item Country Roads 20, 32, 79, and 107
    \item Regional commuter rail stations (Salisbury Mills/Cornwall)
    \item Inactive railroads beds, which are of particular interest as 
    potential rail trail recreation resources, are illustrated on the ~\nameref{map:townzoning} 
    and found in quadrants A2, B1, B2, and C2.
\end{itemize}

\subsection*{Important municipal structures}
\begin{itemize}
    \item Cornwall Town Hall; Cornwall-on-Hudson Village Hall
    \item Cornwall Central School District buildings
    \item U.S. Post Offices
    \item St. Luke’s Cornwall Hospital
    \item Town and village police departments
    \item Town and village fire houses and EMS
    \item Schools and local museums 
\end{itemize}

\subsection*{Surface Water Features}
\begin{itemize}
    \item Lakes and ponds
    \item Beaver Dam Lake
    \item Upper Reservoir
    \item Alec Meadow Reservoir
    \item Arthur’s Pond
    \item Sphagnum Pond
    \item Sutherland Pond
    \item Creeks and brooks
    \begin{itemize}
        \item Moodna Creek
        \item Woodbury Creek
        \item Baby Brook
    \end{itemize}
\end{itemize}

\subsection*{Major landmarks}
\begin{itemize}
    \item Municipal boundaries
    \item Hamlets
\end{itemize}
\includepdf[pages=-,fitpaper]{cornwall_maps/BaseMap.pdf}\label{map:basemap}
\includepdf[pages=-,fitpaper]{cornwall_maps/AerialImagery_2016.pdf}\label{map:aerialimagery2016}
\label{Aerial Imagery 2016}
\includepdf[pages=-,fitpaper]{cornwall_maps/AerialImagery_2017.pdf}\label{map:aerialimagery2017}
\label{Aerial Imagery 2017}