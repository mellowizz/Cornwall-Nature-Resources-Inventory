\subsection*{Why you need this map}
Cornwall has a rich recorded history stretching back to the voyage of Henry 
Hudson up the river named for him in 1609. Much of that history has been 
preserved and celebrated in the area, and Cornwall boasts more houses and 
landmarks on the National Register of Historic Places than any other 
municipality in Orange County. From its founding, through the War of 
Independence, and through the intense expansion of the 1800's, which saw 
Cornwall Landing become a major port for Hudson River commerce, Cornwall has 
played an important role in the history of the Hudson Valley region. \textbf{The 
historical significance of Cornwall is important to regional tourism and local 
property values and is an important resource to protect.}

\subsection*{Historic Resources}\label{subsec:historic}
The area of modern day Cornwall was originally inhabited by the Waoraneck tribe 
who were a Munsee-speaking subgroup of the Lenape (Delaware) nation of native 
peoples. The original name of the Hudson River is \textit{M’hikanituk}, which is 
pronounced mough-hee-kan-i-tuck. \textit{Mough} means ``greatest of all'', 
\textit{heekan} means ``arm of the sea'', or estuary, and \textit{tuck} means
``a river that flows both 
ways''~\citep{stonypointcenter}. The Lenape were one of many nations that made 
up the Algonquin peoples of the Northeast woodlands.
\par
The official title of first Europeans in Cornwall belongs to the MacGregorie 
party who settled in the area in 1685 around the mouth of the Moodna Creek, 
then known as the Waoraneck after the local tribe and later named Murderer's 
Creek by European arrivals. Members of this group established a trading post 
south of the Moodna on Sloop Hill within Cornwall's modern-day boundaries. In 
the ensuing 50 years, English and Scotch families came to the fertile plateau 
above the river meadows naming it "New Cornwall" because of the marked 
similarity to the County of Cornwall, England. During this time farms and 
livestock operations spread throughout the area and Cornwall became a supplier 
of milk, meat and produce that was shipped by sailing barges down the river to a 
rapidly growing New York City.
\par
During the War of Independence, the Continental Army traveled along the roads 
of the hamlets that made up New Cornwall from West Point to Newburgh, and 
General George Washington was known to stop and visit David Sands and other 
friends during that period~\citep{townofcornwall}. \textbf{The Sand’s Ring Homestead,
the David Sutherland House, and the Cornwall Friends Meeting House remain vivid 
reminders of the colonial period in our town}. In 1788 Orange County was subdivided
into numerous townships, thus officially creating the town of ``New Cornwall''. 
The town's name was subsequently changed to ``Cornwall'' in 1797.
\par
The early 1800's saw rapid development of the Cornwall waterfront, and Cornwall 
Landing became a hamlet unto itself. The transportation of coal brought by rail 
from Pennsylvania and other industry like brickworks and lumberyards lead to a 
bustling waterfront that would be unrecognizable to today's residents used to 
the green spaces of Donahue Memorial Park. The shipwreck sheltering the 
entrance to the Cornwall Yacht Club and pilings from the old coal dock, which 
burned in the early 1900's are visible reminders along the riverfront of its 
industrial past.
\par
In the late 1800's Cornwall became popular as a health retreat. Up until the 
early 20$^{th}$ century, city folk flocked to the Hudson Valley region to experience 
the therapeutic powers they believed it to hold. The mountains, fresh air and 
evergreen forests were thought to offer the perfect conditions for good health 
and they were not far from the city. Cornwall was especially popular, offering 
numerous boarding houses and many conveniences of the day, including 
accessibility to the railroad and steamboats, as well as a telegraph office and 
large library~\citep{ruttenber1881}. \textbf{Former boarding houses such as the 
Samuel Brooks House on Pleasant Hill Rd. and the Walter Hand and Patrick Piggot
Houses on Angola Road are just some of the Cornwall homes from this period on 
the Historic Register.}
\par
More recently, Cornwall became the birthplace of the modern environmental 
movement when a plan by Con Edison in 1962 to build a pumped storage 
hydroelectric plant on Storm King Mountain was blocked. This campaign, led by 
The Scenic Hudson Preservation Conference, and the ensuing legal decision set 
new precedence for environmental activism and law, and became the basis for the 
National Environmental Policy Act (NEPA) and New York's State Environmental 
Quality Review Act (SEQRA), regulations which have protected the environment 
for decades since.

\includepdf[pages=-,fitpaper]{cornwall_maps/HistoricandCulturalResources.pdf}\label{map:historicandculturalresources}
\subsection*{Scenic Resources}\label{subsec:scenic}
Cornwall is well known for its scenic vistas and rural character, even as it 
continues to evolve into a bedroom community for New York City. Storm King 
Mountain and its trails have drawn visitors to Cornwall for centuries. This map 
shows the land area encompassing the mountain, the surrounding hills to the 
south in Highland Falls, and a large part of the Village of Cornwall-on-Hudson 
that form part of the Hudson Highlands Scenic Area. \textbf{These lands are 
considered by the New York State Department of Environmental Conservation as 
scenic areas of statewide significance.} Scenic areas of statewide significance 
are areas defined by the New York State DEC that possess unique, highly scenic 
landscapes accessible to the public and recognized for their outstanding 
quality.
\par
To the west, the end of the Schunnemunk ridge descends down into a valley that 
allows the Moodna Creek to flow eastward and then southward into Mountainville. 
This is where one of Cornwall’s most iconic scenic resources, the Moodna 
Viaduct, crosses north to south over the Moodna as well as Orrs Mills and 
Otterkill Roads carrying Metro North commuter trains between New Jersey and 
Port Jervis to the Northwest. The Moodna Viaduct trestle, which was constructed 
between 1904 and 1908 by the Erie Railroad, spans the valley for 3,200 feet and 
is 193 feet high at its highest point, making it the highest and longest 
railroad trestle east of the Mississippi River~\citep{moseronline}. The Moodna 
Valley and Viaduct are considered a scenic area of countywide significance, 
as identified in the 2014 Orange County Open Space Plan. 
\par
Not listed as a statewide or countywide scenic area, but no less iconic and 
vital to local tourism and regional environmental conservation is Schunnemunk 
Mountain. The northern portion of the mountain ridge falls within the borders 
of the Town of Cornwall and runs southwest to its highest point in the 
neighboring town of Blooming Grove. At 1,664-feet, Schunnemunk Mountain is the 
highest mountain in Orange County and offers stunning views of Cornwall to the 
east, the Wallkill River valley to the north, and beyond that the Shawangunk and 
Catskill mountains to the northwest. Due to its height and length, Schunnemunk 
can be seen from much of the rest of Orange County and nearby areas.