This Appendix refers to the Zoning and Parcels Maps for the Town of Cornwall and the Village of Cornwall-on-Hudson. Brief descriptions of permitted uses within each district as well as the natural resources contained therein are bulleted below.

Town of Cornwall
\begin{itemize}
    \item Generally, the majority of districts allow varying levels of residential development.
    \item The Agricultural Rural Residence District permits uses related to agriculture, uses generally appropriate for large open spaces, and single-family detached homes.
    \begin{itemize}
        \item The ARR district is underlain by aquifers in the flatter lands around Schunnemunk Mountain (see A2/A3/B2/B3 in this map and in Public Wells, Aquifers, and Risk Sites Map).
        \item Steep slopes are found at the base of Schunnemunk Mountain (see A3/B3 and in the Steep Slopes Map), between Long Hill Drive and Mineral Springs Road (C4 in the Steep Slopes Map), and at the base of Black Rock Forest (C4 in the Steep Slopes Map).
        \item Along Cornwall’s southeastern border, stepping stone, locally significant, and regionally significant forests are part of this district.
        \item Federally and state designated wetlands are found in this district. Also present are hydric soils, which are probable wetlands. (See grid boxes A2/A3, B2, C4, D2; also see same grid boxes in Wetlands and Hydric Soils Map.)
    \end{itemize}
    \item The Local Shopping District permits a variety of uses, including gasoline service facilities by special permit and uses accessory to motor vehicle repair shops and used vehicle sales.
        \item Of the 3 areas where these districts are found, only the district in Mountainville, where Pleasant Hill Road and state route 32 intersect, presents a concern as this area is underlain by an aquifer, is adjacent to Class C trout spawning stream, and falls within a 100-year and 500-year floodplain.
    \item No concerns for uses permitted in the General Commercial Shopping District are identified as related to the Town’s natural resources.
    \item The Highway Commercial District permits uses ranging from small-scale to large-scale commercial development, including automobile repair facilities, undertaking and funeral establishments, parking lots and parking facilities, large retail facilities, and gasoline service facilities.  Use groups A thru J allow for high maximum development coverage, resulting is a large portion of impervious coverage.
    \begin{itemize}
        \item The District located along federal highway 9W northeast of Willow Avenue is located within a stepping stone forest and regional forest linkage zone.
        \item The District located on state route 32 near the Town of New Windsor boarder is underlain by an aquifer, falls within a regional forest linkage zone, and is adjacent to or within federally and state designated wetlands.
        \end{itemize}
    \item The Mountain and Conservation Residence District permits appropriate uses with low maximum development coverage.  Of the five areas zoned MCR, four fall entirely or partially within protected open space.
    \begin{itemize}
        \item The area bounded by Orrs Mills Road, state route 32, Pleasant Hill Road, and interstate 87 (B2/B3 \& C2/C3) is underlain by aquifers, contains wetlands of federal and state importance and hydric soils, has steep slopes in excess of 15\%, has a Class C stream supporting trout, prime farmland soil, and is a regional forest linkage zone with stepping stone forest.
        \item The area west of interstate 87 and north of the Schunnemunk State Park boundary has regionally significant forest, some prime farmland, and a Class C stream supporting trout.
        \end{itemize}
    \item The Mixed Residence District permits a variety of uses, including single-family dwellings, multiple dwelling development, and clustered higher density residential development (two-family detached, townhouses, row). Use groups C and D permit high maximum development coverage.
    \begin{itemize}
        \item The MR area in B1/B2, which is largely occupied by Cornwall Central High School, is underlain by aquifers, contains hydric soils and federally and state designated wetlands, a Class C stream supporting trout, and contains stepping stone forest within the regional forest linkage zone.
        \item The MR area in A2 contains hydric soils and federally designated wetlands as well as stepping stone forest within the regional forest linkage zone.
        \end{itemize}
    \item The Planned Commercial District permits many uses, including laboratories and related offices, light manufacturing, industrial parks, printing plants, and general manufacturing and industrial processing operations. These uses are allowed a high maximum development coverage.  The PCD areas below are identified by the map grids in which they fall.
    \begin{itemize}
        \item The area in D1, occupied by New York Military Academy, is adjacent to Idlewild Creek—a Class C stream supporting trout.
        \item The area in C1/C2 is the former site of the Firthcliffe Carpet Mill Company and Majestic Weaving, and current site of a number of small businesses. This area is adjacent to the Moodna Creek, a 100-year floodplain Class C stream supporting trout. It is currently designated a DEC remediation site and contains stepping stone forest within the regional forest linkage zone.
        \item The area in B3, occupied by Tectonic Engineering \& Surveying Consultants, is underlain by an aquifer but has a relatively small percentage of impervious coverage.
        \item The area in B4 is completely underlain by an aquifer; includes the Woodbury Creek’s, a Class C trout spawning stream, 100- and 500-year floodplain; a federal wetland and hydric soils; prime farmland soils; and stepping stone forest.  The majority of this area is protected by a conservation easement.
        \end{itemize}
    \item The Planned Industrial/Office District permits many uses, including laboratories and related offices, light manufacturing, industrial parks, printing plant, outdoor storage of painting supplies, raw materials, fuels; general manufacturing and industrial processing operations; and above ground storage of crude oil and volatile products.  Use groups A thru G are allowed a high maximum development coverage.  The PIO areas below are identified by the map grid sections in which they fall.
    \begin{itemize}
        \item The area in D1 is underlain by an aquifer; includes the Moodna Creek’s, a Class C trout spawning stream, 100-year floodplain and a federal wetland.
        \item The area in B1/C1 is almost entirely underlain by an aquifer; contains federally and state designated wetlands, and contains stepping stone forest within the regional forest linkage zone.  Review of the Google Maps satellite image appears to show a sufficient buffer between the existing construction and adjacent wetlands. Development along Hollaran Road does not follow cluster development principles.
        \item The area in B4 is underlain by an aquifer; includes the Woodbury Creek’s, a Class C trout spawning stream, 100- and 500-year floodplain; contains hydric soils; prime farmland soil; and stepping stone forest.
        \end{itemize}
    \item The Planned Residential District permits limited uses. This area in C1 is peppered with federal wetlands and contains stepping stone forest within the regional forest linkage zone. Conventional suburban development abuts the southwestern portion of this pristine patch of forest.
    \item The Suburban Low-Density Residence District permits a variety of uses, including single-family dwellings.
    \begin{itemize}
        \item The area on both sides of Angola Road has a federal aquifer and hydric soils on the northern end, three Class C streams supporting trout, steep slopes in excess of 15\%, and contains stepping stone and regionally significant forests.
        \item The area north of Orrs Mills Road and west of interstate 87 is underlain by a large aquifer, has a number of federally and state designated wetlands and hydric soils, two Class C streams supporting trout, and stepping stone forest in a regional forest linkage zone.
        \end{itemize}
    \item The Suburban Residence 1 District permits uses related to outdoor recreation, agriculture, institutional, and low to medium density housing.
    \begin{itemize}
        \item The area around Beaverdam Lake contains federal wetlands and hydric soils, a Class C stream supporting trout, and stepping stone forest in a regional forest linkage zone.
        \item The area from B1/B2 to D1/D2 contains a number of federal and state wetlands and hydric soils, a number of Class C streams and Idlewild Creek—a Class C trout spawning stream, stepping stone forest in a regional forest linkage zone, and locally significant forest.
        \end{itemize}
    \item The Suburban Residence 2 District permits uses largely related to small professional offices and low to medium density housing. The area in C2/D2 contains a Class C streams and Idlewild Creek—a Class C trout spawning stream, some prime farmland, and a small portion falls within a 500-year floodplain. Use groups D, G, H, and I have high maximum development coverage. 
    \item The Schunnemunk Agricultural Scenic Overlay is located west of interstate 87 and south of Moodna Creek and Orrs Mills Road. Clustered subdivision layout can be required by the Planning Board through a Conservation Subdivision Design Layout, with a minimum open space allocation of 50\%.  The Overlay terms slopes in excess of 30\% as a significant barrier to development.  Areas of 50-100 feet buffer waterbodies, waterways, and wetlands from development.
    \begin{itemize}
        \item Within the Overlay is found 100- and 500-year floodplains, federal and state wetlands and hydric soils, aquifers, prime farmland, steep slopes in excess of 15\%, and regionally significant forest within a regional forest linkage zone.
        \end{itemize}
    \item The Ridge Preservation Overlay is designed to protect the visual and aesthetic resources of the Schunnemunk Mountains and Hudson Highlands ridgelines. Planning Board review includes consideration of plantings of “appropriate native deciduous and/or evergreen vegetation.”
\end{itemize}
Village of Cornwall-on-Hudson
\begin{itemize}
    \item Conservation Residential Districts 1-3 and Suburban Residential District permit single-family cluster development, which was authorized, among other intents, “to preserve the natural and scenic qualities of open space and to protect local ecology, major stands of trees, steep slopes, geological features, and other areas of environmental value” through the flexibility in design and development of land.
    \item The Conservation Residential CR-1 District permits residential, recreational, and riverine uses, with a maximum permitted lot coverage pf 15\%.
    \begin{itemize}
        \item The district is partially underlain by an aquifer and contains a Class C streams supporting trout.
        \end{itemize}
    \item The Conservation Residential CR-2 District (rural) permits residential uses as well as uses compatible with residential and small scale agricultural development, with a maximum permitted lot coverage pf 10\%. 
    \begin{itemize}
        \item The district contains hydric soils and federally designated wetlands, a significant area of very steep slopes in excess of 25\%, and is almost entirely covered by locally significant forest.
        \end{itemize}
    \item The Conservation Residential CR-3 District (scenic) permits residential uses as well as uses compatible with residential development, with a maximum permitted lot coverage of 10\%.  The district includes a conservation green belt setback of 25 feet along both sides of Deer Hill Road where no tree cutting, construction, or other development is permitted.  No criteria for bringing properties into conformance is identified.
    \begin{itemize}
        \item The district contains a Class C stream supporting trout and is covered by locally significant forest.
    \end{itemize}
    \item The Industrial District permits uses such as laboratory, manufacturing, printing, and high-density housing.
    \begin{itemize}
        \item This district is underlain by an aquifer, contains a few federally designated wetlands, is located entirely on a 100-year floodplain, and includes steep and very steep slopes on the southwestern edge of the district. By 2100, a 3-foot sea level rise would inundate over half of the district and a 6-foot rise would result in the inundation the entire area.
        \end{itemize}
    \item The Suburban Residential District permits a variety of uses, including residential, recreational, and instructional.
    \begin{itemize}
        \item The district contains areas within the 100- and 500-year floodplains, a Class C stream supporting trout, locally significant forest, and prime farmland soil. The district contains 5 petroleum bulk storage facilities on Hudson Street and state route 218, one of which is located on a portion of route 218 that flooded during Hurricane Irene.
        \end{itemize}
    \item The Waterfront Recreation District permits riverine uses. The district is entirely underlain by an aquifer and includes an existing petroleum bulk storage facility, contains numerous federally designated wetlands, falls within a 100- and 500-year floodplain, includes prime farmland, and contains a Class C stream supporting trout. By 2100, a 3-foot sea level rise would inundate over half of the district and a 6-foot rise would inundate the entire area.
   \item The Central Business \& Shopping District includes Hudson Street up to River Street as well as streets on either side of Hudson and permits largely standard commercial uses. Any natural resources related to this district are mentioned in the SR District section above.
    \item The View Preservation District Overlay encompasses districts north of Hudson Street and state route 218. The overlay provides for the preservation and protection of Hudson River views from existing public roads, parks, or legally accessible public property under the Village’s Scenic Resources Protection Law. The Law declares the protection of these views for present and future generations in the broader public interest. The District Overlay ensures that use and development on private lands of structures and natural plantings do not impact the views from public areas. Any natural resources related to this district are mentioned in the SR, CR-1, I, and WR district sections above.
\end{itemize}